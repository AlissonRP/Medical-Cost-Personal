% Options for packages loaded elsewhere
\PassOptionsToPackage{unicode}{hyperref}
\PassOptionsToPackage{hyphens}{url}
%
\documentclass[
]{article}
\usepackage{amsmath,amssymb}
\usepackage{lmodern}
\usepackage{ifxetex,ifluatex}
\ifnum 0\ifxetex 1\fi\ifluatex 1\fi=0 % if pdftex
  \usepackage[T1]{fontenc}
  \usepackage[utf8]{inputenc}
  \usepackage{textcomp} % provide euro and other symbols
\else % if luatex or xetex
  \usepackage{unicode-math}
  \defaultfontfeatures{Scale=MatchLowercase}
  \defaultfontfeatures[\rmfamily]{Ligatures=TeX,Scale=1}
\fi
% Use upquote if available, for straight quotes in verbatim environments
\IfFileExists{upquote.sty}{\usepackage{upquote}}{}
\IfFileExists{microtype.sty}{% use microtype if available
  \usepackage[]{microtype}
  \UseMicrotypeSet[protrusion]{basicmath} % disable protrusion for tt fonts
}{}
\makeatletter
\@ifundefined{KOMAClassName}{% if non-KOMA class
  \IfFileExists{parskip.sty}{%
    \usepackage{parskip}
  }{% else
    \setlength{\parindent}{0pt}
    \setlength{\parskip}{6pt plus 2pt minus 1pt}}
}{% if KOMA class
  \KOMAoptions{parskip=half}}
\makeatother
\usepackage{xcolor}
\IfFileExists{xurl.sty}{\usepackage{xurl}}{} % add URL line breaks if available
\IfFileExists{bookmark.sty}{\usepackage{bookmark}}{\usepackage{hyperref}}
\hypersetup{
  pdftitle={Uma continuação da breve análise sobre o PIB per capita dos Estados Brasileiros no ano de 2019},
  pdfauthor={Alisson Rosa e Vítor Pereira},
  hidelinks,
  pdfcreator={LaTeX via pandoc}}
\urlstyle{same} % disable monospaced font for URLs
\usepackage[margin=2cm]{geometry}
\usepackage{longtable,booktabs,array}
\usepackage{calc} % for calculating minipage widths
% Correct order of tables after \paragraph or \subparagraph
\usepackage{etoolbox}
\makeatletter
\patchcmd\longtable{\par}{\if@noskipsec\mbox{}\fi\par}{}{}
\makeatother
% Allow footnotes in longtable head/foot
\IfFileExists{footnotehyper.sty}{\usepackage{footnotehyper}}{\usepackage{footnote}}
\makesavenoteenv{longtable}
\usepackage{graphicx}
\makeatletter
\def\maxwidth{\ifdim\Gin@nat@width>\linewidth\linewidth\else\Gin@nat@width\fi}
\def\maxheight{\ifdim\Gin@nat@height>\textheight\textheight\else\Gin@nat@height\fi}
\makeatother
% Scale images if necessary, so that they will not overflow the page
% margins by default, and it is still possible to overwrite the defaults
% using explicit options in \includegraphics[width, height, ...]{}
\setkeys{Gin}{width=\maxwidth,height=\maxheight,keepaspectratio}
% Set default figure placement to htbp
\makeatletter
\def\fps@figure{htbp}
\makeatother
\setlength{\emergencystretch}{3em} % prevent overfull lines
\providecommand{\tightlist}{%
  \setlength{\itemsep}{0pt}\setlength{\parskip}{0pt}}
\setcounter{secnumdepth}{5}
\usepackage[brazil]{babel}
\usepackage{booktabs}
\usepackage{longtable}
\usepackage{array}
\usepackage{multirow}
\usepackage{wrapfig}
\usepackage{float}
\usepackage{colortbl}
\usepackage{pdflscape}
\usepackage{tabu}
\usepackage{threeparttable}
\usepackage{threeparttablex}
\usepackage[normalem]{ulem}
\usepackage{makecell}
\usepackage{xcolor}
\ifluatex
  \usepackage{selnolig}  % disable illegal ligatures
\fi

\title{Uma continuação da breve análise sobre o PIB per capita dos Estados Brasileiros no ano de 2019}
\author{Alisson Rosa e Vítor Pereira}
\date{}

\begin{document}
\maketitle

{
\setcounter{tocdepth}{2}
\tableofcontents
}
\section{Análise de Influência - DFBETAS}

No diagnóstico dfbetas, que informam o grau de influência que a observação \(i\) tem sobre o coeficiente \(\hat{x_i}\), ou seja, sobre os parâmetros \(\hat{\beta_i}\). Então essa é uma medida completar ao DFFITS, que verifica a influência de \(i\) tem sobre o valor seu próprio valor ajustado \(\hat{y_i}\).

\subsection{DFBETAS - Modelo Original}

Nesta seção teremos as análises dos DFBETAS para o modelo inicial com o DF.

Nos gráficos abaixo podemos ver que o DF está fora de todos os limites do DFBETAS, principalmente no DFBETA do x4 em que realmente o DF achata muito o gráfico. Os estados de AM, RJ e MS são candidatos a ponto de influente nas covariávies de Área, Densidade Demográfica e Pobreza.

\begin{center}\includegraphics{extra_files/figure-latex/unnamed-chunk-1-1} \end{center}

\begin{center}\includegraphics{extra_files/figure-latex/unnamed-chunk-1-2} \end{center}

\begin{center}\includegraphics{extra_files/figure-latex/unnamed-chunk-1-3} \end{center}

\subsection{DFBETAS - Modelo Ajustado}

Nesta seção teremos as análises dos DFBETAS para o modelo inicial sem o DF.

Nos gráficos abaixo podemos ver que não tem um estado que está fora de todos os limites do DFBETAS. MS está fora do DFBETA 1 e 5, SP está fora do DFBETA 1 e 2, o AM está fora do DFBETA 3 e RJ fora do DFBETA 4. Também podemos notar que nenhum dos possíveis pontos influentes achata o gráfico de uma forma extrema como acontecia com o DF no modelo original.

\begin{center}\includegraphics{extra_files/figure-latex/unnamed-chunk-2-1} \end{center}

\begin{center}\includegraphics{extra_files/figure-latex/unnamed-chunk-2-2} \end{center}

\begin{center}\includegraphics{extra_files/figure-latex/unnamed-chunk-2-3} \end{center}

\section{Análise Descritiva}

Nesta seção veremos um breve resumo das variáveis de estudo, com medidas descritivas, medidas de correlação do banco original com o DF e a comparação dessas medidas com o banco e o modelo ajustado.

Começaremos por uma tabela resumo, com informações sobre as covariáveis, em comparação com a banco ajustado podemos notar uma diferença significativa matematicamente da média do PIB per capita 27172,676 no banco com DF e 25121,530 no banco ajustado, o desvio padrão no banco original é 46\% maior, assim como podemos notar um aumento da média na Densidade Demográfica de 30\% e 48\% no desvio padrão, com apenas uma observação a mais.

\begin{table}[H]

\caption{\label{tab:unnamed-chunk-3}Resumo das variáveis: }
\centering
\begin{tabular}[t]{l|c|c|c|c|c|c}
\hline
  & n & Média & Desvio Padrão & Mediana & Minímo & Máximo\\
\hline
PIB & 27 & 27172,676 & 14336,816 & 22936,280 & 12788,750 & 80502,470\\
\hline
IDHe & 27 & 0,718 & 0,050 & 0,717 & 0,636 & 0,828\\
\hline
Área & 27 & 315215,579 & 375101,871 & 224273,831 & 5760,783 & 1559168,117\\
\hline
Densidade Demográfica & 27 & 75,917 & 120,552 & 36,090 & 2,660 & 523,410\\
\hline
Pobreza & 27 & 0,116 & 0,075 & 0,128 & 0,017 & 0,263\\
\hline
\end{tabular}
\end{table}

A título de curiosidade vejamos o PIB por região:

\begin{table}[H]

\caption{\label{tab:unnamed-chunk-4}PIB per capita por Região}
\centering
\begin{tabular}[t]{c|c|c|c}
\hline
Região & Média PIB & Desvio padrão & Estados\\
\hline
Centro-oeste & 45561 & 23649 & 4\\
\hline
Norte & 21049 & 2643 & 7\\
\hline
Nordeste & 16358 & 2070 & 9\\
\hline
Sul & 38062 & 1328 & 3\\
\hline
Sudeste & 35667 & 9566 & 4\\
\hline
\end{tabular}
\end{table}

Notamos que apenas o acréscimo do DF no banco de dados, faz com que a posição do Centro-Oeste de PIB per capita, saia do 3° lugar em média para o 1°, sendo a única região com maior de 40 mil. Assim como é a região disparadamente com maior desvio padrão, sendo 147\% maior que o desvio padrão da região sudeste, a região com segundo maior desvio padrão.

\begin{center}\includegraphics{extra_files/figure-latex/unnamed-chunk-5-1} \end{center}

\subsection{Correlação}

Nesta seção, será comparada a correlação dos banco de dados com o DF e sem o DF.

Temos que a matriz de correlação do banco original é:

\begin{table}[H]

\caption{\label{tab:unnamed-chunk-6}Correlação entre as variáveis}
\centering
\begin{tabular}[t]{l|c|c|c|c|c}
\hline
  & PIB & IDHe & Área & Densidade Demográfica & Pobreza\\
\hline
PIB & 1,000 & 0,773 & -0,112 & 0,727 & -0,759\\
\hline
IDHe & 0,773 & 1,000 & -0,003 & 0,405 & -0,716\\
\hline
Área & -0,112 & -0,003 & 1,000 & -0,371 & 0,193\\
\hline
Densidade Demográfica & 0,727 & 0,405 & -0,371 & 1,000 & -0,380\\
\hline
Pobreza & -0,759 & -0,716 & 0,193 & -0,380 & 1,000\\
\hline
\end{tabular}
\end{table}

Temos que a matriz de correlação do banco ajustado é:

\begin{table}[H]

\caption{\label{tab:unnamed-chunk-7}Correlação entre as variáveis}
\centering
\begin{tabular}[t]{l|c|c|c|c|c}
\hline
  & PIB & IDHe & Área & Densidade Demográfica & Pobreza\\
\hline
PIB & 1,000 & 0,822 & 0,016 & 0,391 & -0,880\\
\hline
IDHe & 0,822 & 1,000 & 0,058 & 0,237 & -0,692\\
\hline
Área & 0,016 & 0,058 & 1,000 & -0,376 & 0,158\\
\hline
Densidade Demográfica & 0,391 & 0,237 & -0,376 & 1,000 & -0,292\\
\hline
Pobreza & -0,880 & -0,692 & 0,158 & -0,292 & 1,000\\
\hline
\end{tabular}
\end{table}

Podemos notar que a correlação entre IDHe e PIB per capita cresce no modelo ajustado, também invertendo totalmente a relação linear entre a área e a variável desfecho, deixando de ser uma correlação positiva para uma correlação negativa. Como o DF tem a maior densidade demográfica e o maior PIB per capita, a correlação entre as duas variáveis aumenta quase 100\%, assim também diminuindo a correlação da taxa de pobreza extrema e a variável resposta, ficando mais evidente a influência do Distrito Federal na análise.

\section{Modelos de Regressão Linear}

Tem-se portanto como resumo do modelo final com o DF como observação é a seguinte tabela:

\begin{table}[H]

\caption{\label{tab:unnamed-chunk-8}Resumo do modelo original}
\centering
\begin{tabular}[t]{c|c|c|c}
\hline
Coeficientes & Estimativa & Erro Padrão & p-valor\\
\hline
(Intercept) & -29249,678 & 24898,553 & 0,253\\
\hline
IDHe & 81293,014 & 32880,246 & 0,022\\
\hline
Área & 0,006 & 0,003 & 0,075\\
\hline
`Densidade Demográfica` & 62,199 & 10,675 & 0,000\\
\hline
Pobreza & -73736,452 & 21224,661 & 0,002\\
\hline
\end{tabular}
\end{table}

Que informa que o intercepto não significativos a 5\% e \(R^2\) dado por 0.876 que informa que aproximadamente 87.6\% da variação do PIB per capita dos estados é explicada pelas covariáveis propostas.

E o modelo final sem o DF é dado pela tabela abaixo:

\begin{table}[H]

\caption{\label{tab:unnamed-chunk-9}Resumo do modelo final}
\centering
\begin{tabular}[t]{c|c|c|c}
\hline
Coeficientes & Estimativa & Erro Padrão & p-valor\\
\hline
(Intercept) & -18296,206 & 15527,248 & 0,252\\
\hline
IDHe & 70391,625 & 20444,077 & 0,002\\
\hline
Área & 0,004 & 0,002 & 0,046\\
\hline
`Densidade Demográfica` & 22,830 & 9,292 & 0,023\\
\hline
Pobreza & -80336,960 & 13190,734 & 0,000\\
\hline
\end{tabular}
\end{table}

Portanto, podemos notar, novamente, a influência do DF, em que temos uma variação de 60\% no intercepto, uma diminuição do beta 2 em 15\%, uma pequena alteração no beta 3, uma diminuição de mais de 180\% no beta 4 e um aumento da influência negativa de quase 10\% no beta 5. Sendo assim, o modelo ajustado com DF é consideralvelmente diferente do modelo sem DF.
Desse modo notamos um aumento, também, na explicação geral das covariáveis no PIB per capita pelo modelo final do projeto, pois o \(R^2\) no modelo com DF é por 0.876, e no modelo sem o DF é 0.899, mesmo com uma observação a menos o modelo proposto pelo trabalho é superior em explicação da variável desfecho.

\section{Conclusão}

Podemos notar que o modelo proposto cumpre todos os testes de pressupostos, ao contrário do modelo original, tendo covariáveis que fazem sentido com realidade para explicar o PIB per capita, sendo esses de áreas realmente interessantes para o estudo desse componente social, como: População, Geografia, Condição de Vida e Educação.

Dessa maneira percebemos a necessidade da remoção da observação do DF pelas análises de influência, análise descritiva e análise dos modelos ajustados, verificando uma clara interferência dessa observação no modelo final, que está bem justificada na parte principal do trabalho. Podemos confirmar que essa é uma análise de regressão acurada, pois possui covariáveis que explicam a variável dependente de forma significativa e correlata com os conhecimentos de outras áreas, uma explicação da variável resposta com quase 90\% pelas covariáveis, explicação altíssima.

\end{document}
